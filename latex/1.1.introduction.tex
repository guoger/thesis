\chapter{Introduction}
It has been proven that the uses of Information and Communication Technology (ICT) can effectively support poverty alleviation and livelihood enhancement\cite{XXX}. Bringing ICT in developing counties and under-served areas requires multinational partnerships and unique research commitments. There exist many outstanding projects and business cases in the field, although not all of them is reproducible, especially when adapted to a different culture and work environment.
Two districts have been selected to conduct a pilot project, namely Serengeti Broadband Network, aiming to enhance ICT penetration, build buying power and establish a foundation of rural ICT development. In this thesis report, we cover: 1)A brief history and background of project; 2)Current states and challenges; 3)Our approaches and implementation; 4)Future plan. This report stands as a comprehensive documentation of current states of SBN development.

\section{Background}
ICT4RD\footnote{www.ict4rd.ne.tz} is designed as a research and business development project, aiming at provisons of ICT services in under-served areas of Tanzania. The project is funded by Swedish International Development Agency (SIDA)\footnote{www.sida.se}, and coordinated by Tanzania Commission of Science and Technology (COSTECH)\footnote{www.costech.or.tz}; Dar es Salaam Institute of Technology (DIT)\footnote{www.dit.or.tz} Tanzania; and the Royal Institute of Technology (KTH)\footnote{www.kth.se}, Sweden.Two pilot projects are created under ICT4RD, respectively Serengeti Broadband Network development (SBN-development) and Wami project. In this report, we only focus on SBN project.

SBN development aims at building a self-sustained Local Access Network (LAN) convering two districts\cite{nungu_thesis}. It hosts services locally, such as VoIP, mails. When upper link exists at any site of the LAN, other parts can also be online. By interconnecting schools, dispensaries and governments, a variety of applications are proposed and implemented over the network, such as e-learning, e-governing and e-health.

It is the seventh year of SBN project. During these years, enormous research effort has been contributed from partners to establish a robust, scalable and sustainable broadband network island.
%TODO more of backgournd

\section{Related Work}
There are many successful rural ICT projects which stand as good references. Several of them are presented here, mainly focusing on technical issues while excluding business development.
\textbf{Macha Project}
\textbf{Nipel Wireless}
\textbf{South Africa}

\section{Problem Statement}
ICT development in rural areas is a challenging task which requires innovations on both technical and managerial sides. We identify following main obstacles that limit technology deployment:
\begin{itemize}
\item Low affordability. In rural development, expenditure has alwasy been a critical factor, not only during the phase of procurement, but also maintainance and refurbishment. Necessary trade-off needs to be made between capacity and cost.
\item Poor supply chain, especially power shortage. The lack of electricity is always a limitation in rural development. In Tanzaina, only 14\% of the country is electrified and the figure reduces to 2\% in the case of rural area\cite{XXX}. Innovative power source such as solar is highly desired. Even those sites along the power line are also challenged by frequent power outage and voltage spikes.
\item Harsh environment. In rural sites, temperature is considerably higher than recommanded operational level, especially at those sites where equipments boxes are mounted outdoor.
\end{itemize}

\section{Approach}
%TODO elaborate
Special requirements need to be treated differently with innovative approaches. In SBN development project, we apply iterative approaches to test new solutions. Different components are gradually replaced with newly developed technologies. we carefully select off-the-shell hardware and open source software to reduce the cost and enhance rebustness. To attack the problem of power supply, we run our equipments over heterogeneous power source, including sink device such as battery or super capacitor. By reducing the power consumption, we minimize discharge cycles and prolong battery life.

\section{outline}
Remaining content is organized as following: Chapter 2 provides an overview of previous projects of SBN. It then explains current network topology and administration; Chapter 3 demostrate the design of low power router. We discuss about the testing result in this chapter as well. In chapter 4, we draw our conclusion and propose improvements that could be done in the future.
