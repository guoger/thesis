\chapter{Network Administration}
In rural ICT development, technologies should be affordable and easy to use.
%TODO some preface
\section{An Overview of SBN}
Technology selection in rural ICT development is highly affected by physical environment and the demand of services. Conditions are often different from one site to another, hence not replicable in its entirety. As introduced in section \ref{related_proj}, Macha project represents a typical setting of rural ICT envrionment, while Nipel project serves as a good example of distribution of network. In this section, Four communication technologies are presented and evaluated against SBN environment.

\textbf{Optical fiber links} are favorable due to its capacity and durability. Although deployment and maintainence require special tools and skills, and civil work involved is immense. Building a fiber line in rural area demands innovative cooperation to distibute risk and cost, as well as sharing the benefit. In the case of SBN development, during the time that Tanzania set up the power line between two districts, a 140km optical fiber link was also established along the power transmission line and owned by Tanzanian power company, TANESCO\footnote{http://www.tanesco.co.tz/}. The fiber was donated to ICT4RD in exchange for network connection. To distribute fiber backbone network, Low power routers that support both optical fiber and copper links are developed. More details can be found in section \ref{router}

\textbf{Terrestrial Wireless} is an optimal approach in rural first-mile delivery comparing to tranditional landline communication. It is easier to deploy and highly customizable to adapt to different landform. Among various wireless technologies, IEEE 802.11 family (more commonly known as WiFi) running in license-free 2.4GHz or 5GHz offers satisfactory bandwidth while eliminating the cost of frequency registration. In SBN, WiFi is intensively deployed at first mile to distribute connection from optical fiber link to end users. Furthermore, optical fiber backbone is extended by WiBACK\footnote{www.wiback.org} network which enable decent video streaming at remote sites.

Many rural ICT projects deploy \textbf{Very Small Aperture Terminals} to source the Internet from satellite. This approach is highly favorable for remote sites where establishment of infrastructure is simply not feasible. Although most of satellite link come at a very high price and limited bandwidth. In SBN, TTCL extends their data service to Bunda town, where we link our LAN to the Internet.

There exist novel and untraditional approaches in rural ICT development to link remote villiges, such as \textbf{Delay Tolerent Network} in DakNet\cite{pentland2004daknet}, which may physically transport bulk of data by vehicles. In spite of high bandwidth, latency introduced is not suitable for real-time applications such as video call.

\section{Network Topology and Administration}


\section{Key Challenges}
