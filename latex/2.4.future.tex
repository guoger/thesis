\chapter{Conclusion and Future Work}\label{conclusion}
\section{Conclusion}
Starting from the user case, by applying the concept of divide-and-conquer, we are able to formulate the problem and identify the key challenges. Given the impossibility of achieving consistency and availability at the same time, we propose a multi-master system with reasonable assumptions and compromises. We surveyed a variety of existing technologies and investigate the potential to map them into our case. With extensions and proper configurations, SymmetricDS is tuned to serve our purpose. To achieve low-power consumption and affordability, we test the capacity and proposed a cluster which can be easily scaled out. We build Moodle web service on the cluster with integrated synchronization functionalities. We conclude that this prototype has the potential to serve in rural areas where highly available web services are desired and number of users is limited.

\section{Future Work}
At the beginning, the plan was to implement the system and test it in production environment. Although enormous effort has been put into exploring a variety of technologies. Also the overhead to adapt to local work culture disturbs original plan. Thus, the next step will be testing and debugging the system in a real production service.

All web applications encounter same obstacles when adapted to rural area. From the beginning of design, we aim at a generic solution for database-driven web services. This possibility should be further explored.
